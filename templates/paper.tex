\documentclass{amsart}

% PACKAGES
%%%%%%%%%%

% Essential 
\usepackage[utf8]{inputenc} % encoding 
\usepackage[T1]{fontenc} % also needed for accents, etc. 
\usepackage{amsmath, amssymb, amsthm, amsfonts} % absolute essential for math
\usepackage{mathtools} % extends amsmath, add coloneqq for :=
\usepackage{graphicx} % figures
\usepackage[authoryear]{natbib} % bibliography (author-year style for citations)

% Often useful
\usepackage{cleverref} % *before* hyperref
\usepackage{hyperref} % hyper-links 
\usepackage{dsfont} % mathds{1} for indicator (part of doublestroke package)
\usepackage{algorithm2e} % algorithms

% revision process
\usepackage{todonotes} % as the name suggests

% Books / thesis
\usepackage{glossaries} % as the name suggests
\usepackage{geometry}  % or use e.g. KOMA style
% babel TODO

% MACROS
%%%%%%%%

% TODO

\begin{document}

\begin{abstract}
  This is the abstract. Usually no math here (in doubt, check the journal
  guidelines.)
\end{abstract}

\section{Introduction}\label{sec:intro}

A gentle introduction to stuff.

\section{Theory}%
\label{sec:Theory}

A very important equation:
\[ \int_0^{2\pi} \sin(x) dx = 0. \]

Recall that:
\begin{itemize}
  \item No blank line \emph{before} an equation. 
  \item Function $\sin$ is a math operator, TODO.
\end{itemize}

A multi-line equation:
\begin{align}
  \label{eq:}
  f(x) = 
  
\end{align}


\end{document}

