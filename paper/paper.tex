\documentclass{scrartcl}  % article class from KOMA-Script (A4 compatible)
% amsart is also a good default 

% Essential packages
% ==================

% encodings and fonts
\usepackage[utf8]{inputenc}  % input encoding 
\usepackage[T1]{fontenc}  % font (output) encoding
\usepackage{lmodern}

% math
\usepackage{amsmath}  % main AMS package
\usepackage{amsthm}  % theorems (still need to define theorem envs, see below)
\usepackage{amssymb}  % math symbols, e.g. \mathbb{R}
\usepackage{amsfonts}  % better fonts for math
\usepackage{mathtools}  % extends amsmath (e.g. \eqcolon)

% figures
\usepackage{graphicx}

% bibliography
\usepackage[authoryear]{natbib}
%\usepackage[numbers, square]{natbib}  % use this instead for numbered refs
\bibliographystyle{apalike}  % see natbib doc for other styles 
% Note: the journal will provide its own style

% Recommended packages
% ====================

\usepackage{hyperref}  % hyper-links in PDF
\usepackage{cleveref}  % Clever cross-references; must be loaded *after* hyperref
\usepackage{dsfont}  % \mathds{1} for indicator function
\usepackage{algorithm2e}  % algorithms

% Packages that may be useful in certain cases (uncomment if needed)
% ==================================================================

% PhD thesis, book
%\usepackage{glossaries}  % as the name suggests
%\usepackage[a4paper]{geometry}  % not needed with a KOMA class (scrbook for books)
%\usepackage[french, english]{babel}  % use both languages in the same document

% revision
%\usepackage{todonotes}  % as the name suggests

% Theorems environments (amsthm)
% ==============================
\newtheorem{thm}{Theorem}
\newtheorem{cor}{Corollary}
\newtheorem{prop}{Proposition}
\newtheorem{lem}{Lemma}
\newtheorem{rem}{Remark}

% Macros (don't go crazy!)
% ========================

\newcommand{\R}{\mathbb{R}} % set of real numbers
\newcommand*\dd{\mathop{}\!\mathrm{d}} % d in dx in an integral (+ proper spacing)



\begin{document}

\title{My first paper}
\author{P. H. Dee \and Super Visor}
% \institute{University of Andromeda}
\date{}  % put \today for current date
\maketitle

\begin{abstract}
  This is the abstract. Usually no math here (in doubt, check the journal
  guidelines.)
\end{abstract}

\section{Introduction}\label{sec:intro}

This is the introduction, where we may cite the book of \citet{book_sasha}, or
the paper by \citet{cristina_arxiv}.  We can also do citations in parenthesis
\citep{anna_mirror}.


\section{Main results}\label{sec:results}

No blank line before an equation:
\[ \int_0^{2\pi} \sin(x) \dd x = 0
\]
and no blank line after, unless the equation ends the paragraph. (In that case,
make sure there is a dot at the end of the equation.)

We are very proud of our theorem.
\begin{thm}
  For all $x \in \R$, we have $x^2 > 0$.
\end{thm}
Its proof is in the appendix.

This theorem has a nice corollary:
\begin{cor}
  For all $x \in \R$, $x^4 > 0$.
\end{cor}

\section{Numerical experiments}\label{sec:numerics}

Time to show a nice image; see \cref{fig:boxplot_first}.

\begin{figure}
  \begin{center}
    \includegraphics[scale=0.4]{figs/phd_guilt.jpg}
  \end{center}
  \caption{credits: PhD comics}
  \label{fig:boxplot_first}
  % WARNING: first caption, then label
\end{figure}

% You can also put the bibliography after the appendix
\bibliography{bib/my_biblio}

\appendix

\section*{Proofs}

The proof of our theorem:
\begin{proof}
  The proof is left as an exercise to the astute reader.
\end{proof}

\end{document}

